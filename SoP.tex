\documentclass[a4]{article}
\usepackage[letterpaper,margin=1in]{geometry}
\title{Statement of Purpose: Ver1}
\author{Hiroki Yoshida}
\date{}

\begin{document}
\maketitle

My research interest is strongly focused on to the theoretical study of physics, especially condensed matter physics. Through a research on nonequilibrium statistical mechanics, I realized that the theoretical study very fits to me but I need to do topics which is more relevant to the real world physics and show exotic and fascinating phenomenae. For that point, the field of topological phases of matter was very attractive. Prof. Bernevig is one of the world's leading researcher in that field and hence, learning in Princeton is the best choice for me.\par


% \section{current research summary}
I am doing research on the topological phases of matter under supervision of Prof.Murakami in Tokyo Tech now. My original theme was theoretical understanding of the photovoltaic effect in a van der Waals interface, which was experimentally observed recently. This phenomenon is caused by the breaking of symmetry by stacking two sheets of materials with different rotational symmetries. This "symmetry breaking by symmetry" itself is not only interesting idea from theoretical point of view, but also very important for future applications becase of the varitety of symmetries we can think of. To understand this promising phenomenon, I am trying to construct a model which minimally captures the essence right now. \par

In parrallel to this theme, as a result of my eager to aqcuire research experiences as much as possible, I recently began another research too. It is about the newly synthesized hexagonal boron nitride sheet. By the Brillouin zone folding from that of Graphene, the dirac cone get gapped. I constructed a tight binding hamiltonian of this lattice and confirmed an existence of the edge modes by myself so far. This work is also still in progress but through this ressearch, I aquired an ability to construct a tight binding model for the desired system in both finite and infinite systems and analyze it analytically and numerically. This would be very helpful in research in graduate school and beyond.\par
% \begin{itemize}
%   \item topological band theory
%   \item constructing tight binding hamiltonian
%   \item optical response
%   \item van der waals heterostructure
%   \item Why this field of study
% \end{itemize}

% \section{learnings beyond standard carriculum}
Although I enrolled in to the university by the award for the presentation of my experimental research results on photocatalystic activity of TiO$_2$ which I had been doging thoroughout high school, my interest gradually shifted to the theoretical studies in physics by taking undergraduate physics courses. Since our university puts limitation for the numer credits which can be taken per year, I could not take advanced callses officially. However, my enthusiasm could not be stooped and I directly asked teachers for permission to attend coureses beyond carriculum. As a result, though it was a very busy days, I had finished learning most of the physics courses in the middle of the third year.\par
The most striking thing for me to decide to go on to the theoretical study is the research on nonequilibrium statistical mechanics under Specially appointed Prof.Takahashi in Tokyo Tech. Though it is completely irrelevant with the standard physics carriculum, I had time after taking all the courses and was curious about how research is conducted, I directly asked him to do research in collaboration. This is my first research experience. Ever since the Augst 2020 until now, we are doing research on the dynamical Lee-Yang zeros of the simple two level system. Lee-Yang zeros are zeros of partition function of the system with complex parameters. Though it is not present in the real valued physical parameters, it is very useful to determine phase transitions in the thermodynamic limit and also recently experimentally utilized in a system where physical parameters can effectively considered to be complex. These known results are all for static systems without time dependencies. We are currently tackling the periodically driven case with simple two level systems. In time dependent case, the situationn changes drastically and determination of the zeros itself become very difficult. I mainly do numerical calculations by Python and c++ and also formulated some general formulas for current distribution functions using dynamical Lee-Yang zeros. This research is also still in progress and the results are not yet published, I learned alot of difficult points in theoretical studies. But this 15 month research taught me how perseverance I have toward research in physics and how I am in love with physics.\par
After getting the Ph.D degree, I want to get an academic position in university. As a preparation for that, I actively participated in some activities and reinforced some abilities. One of them is the university official survey team called "student survey". I was a president of that team for more than 2 years and learned how to manage people in the team. For my hard work and prominent leadership, I received a "student leadership award 2021". This leadership and collaborative ability will be my strength in research field too.\par
Princeton is my top destination for graduate school research. The main reason why I chose Princeton is that Princeton physics is highly consentrating on educating independent researchers. Of course, research activities under the legendary teachers in my field such as Prof.Bernevig or Prof.Haldane will be an extraordinary oppotunity and I will definetely make the most of it. However, to be an independent researcher as soon as possible, I think that satisfying with making use of given chances is not enough. I need to collaborate with other young researchers as much as possible. For this in my mind, Princeton is the best platform to do graduate school research.


% \begin{itemize}
%   \item study in advance
%   \item research on dynamical lee-yang zeros
%   \item enthusiasm to theoretical research
% \end{itemize}

% \section{outside of research}
% \begin{itemize}
%   \item working experience as a TA
%   \item president of Student survey $\rightarrow$ leadership award
%   \item co-working capability
% \end{itemize}

% \section{Why Princeton?}
% \begin{itemize}
%   \item Future dream
%   \item what wanna do
%   \item reson why I chose this school
% \end{itemize}

\end{document}
