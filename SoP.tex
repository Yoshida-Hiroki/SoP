\documentclass[a4]{article}
\usepackage[letterpaper,margin=1in]{geometry}
\title{Statement of Purpose: Ver6}
\author{Hiroki Yoshida}
\date{}

\begin{document}
\maketitle

My research interest is strongly focused on the theoretical study of physics, especially condensed matter physics. Through my research on nonequilibrium statistical mechanics, I realized that theoretical study suits me very well. Meanwhile, I have become more interested in topics which are more relevant to real-world physics. With regard to this, the field of topological phases of matter is very attractive to me. Princeton's Prof. Bernevig is one of the most outstanding researchers in the world, and hence, learning at Princeton is the best choice for me.\par

\section{learnings beyond the standard curriculum}
I enrolled into Tokyo Institute of Technology through the presentation of my experimental research results on photocatalytic activity of TiO$_2$, which I had been doing throughout high school. At that time, my interest was mainly phenomenas themselves, and I had no preference in theory. After taking several classes, it gradually shifted to theoretical studies in physics. Since our university puts limitations on the number of credits that can be taken per year, I could not take advanced classes officially. However, my enthusiasm could not be stopped and I personally asked teachers for permission to attend courses beyond the curriculum. As a result, I had finished learning most of the physics courses by the middle of my third year.\par

% especially this block has long long sentences.
The most striking experience that catapulted me into the world of theoretical physics is the research on nonequilibrium statistical mechanics under the gudance of Specially appointed Prof. Takahashi in Tokyo Tech. Though it is not within the standard physics curriculum, I had time after taking all the courses and was curious about how research is conducted, I personally asked him to research with his collaboration. Since August 2020, we have been studying the dynamical Lee-Yang zeros of the simple two-level system. Lee-Yang zeros are zeros of the partition function of a system with complex parameters. Though it is not present in the real-valued physical parameters, it is very useful to determine phase transitions in the thermodynamic limit and also recently experimentally utilized in a system where physical parameters can effectively be considered to be complex. These known results are for static systems without time dependence. Therefore, we are currently tackling the hitherto unknown periodically driven cases. In time-dependent cases, the situation changes drastically and the determination of the zeros itself becomes very difficult. I mainly did numerical calculations by Python and c++ and also came up with some general formulas for current distribution functions using dynamical Lee-Yang zeros. This research is still in progress and the results are not yet published, but not only did I learn a lot about nonequilibrium physics, but I learned a great deal about my perseverance and love for physics too. Combining it with my original interests in phenomena, I have decided to specialize in condensed matter theory.\par

\section{current research}
I am investigating the topological phases of matter under the supervision of Prof. Murakami in Tokyo Tech now. My original theme was the theoretical understanding of the photovoltaic effect in a van der Waals interface, which has been experimentally observed recently. This effect is caused by the breaking of symmetry by stacking two sheets of materials with different rotational symmetries. This "symmetry breaking by symmetries" itself is not only an interesting idea from a theoretical point of view but also very important for future applications due to the variety of symmetries not included yet in the experiments. Like Moir\'{e} crystals, lack of periodicity makes it difficult to understand this phenomenon. I am trying to construct a simple model which minimally captures the essence of this phenomenon. By graduation, I want to further understand the mechanism of this and hopefully publish the results. \par

Due to my enthusiasm to acquire as much research experience as possible, I have recently begun another research project. This project deals with newly manufactured hexagonal boron-carbon-nitride sheet. By the Brillouin zone folding from that of the Graphene and change in symmetry, the Dirac cones get gapped and localized states appear in the finite system. I constructed a tight binding hamiltonian of this lattice and confirmed the existence of the edge modes thus far. This work is also still in progress, but through this research, I acquired the ability to construct a tight-binding model for the desired system in both finite and infinite systems and analyze it analytically and numerically. I believe this would be very helpful in my research in graduate school and beyond.\par
% \begin{itemize}
%   \item topological band theory
%   \item constructing tight binding hamiltonian
%   \item optical response
%   \item van der waals heterostructure
%   \item Why this field of study
% \end{itemize}


\section{future plan}
After getting a Ph.D. degree, I want to get an academic position at a university. As a preparation for that, I have actively participated in activities apart from physics research. One of them is the university official survey team called "student survey". I was honored to be the president of that team for more than 2 years (currently vice president) and learned how to work, manage, and lead people in a team environment. For my hard work, I received a "student leadership award 2021". My leadership and collaborative skills will undoubtedbly be a strength for me in the research field too.\par

Princeton is by far my first choice for graduate school research. The main reason why I chose Princeton is that Princeton physics highly focuses on educating independent researchers. It goes without saying, research activities under the legendary teachers in my field such as Prof. Bernevig and Prof. Haldane will be an extraordinary opportunity and I will absolutely make the most of it. However, to be an independent researcher who can establish new fields of study, being satisfied with only making use of given chances is simply not enough. Proactive research collaboration with young researchers in a good environment is essential. Keeping all of this in my mind, Princeton is the best platform to do graduate school research for me.


% \begin{itemize}
%   \item study in advance
%   \item research on dynamical lee-yang zeros
%   \item enthusiasm to theoretical research
% \end{itemize}

% \section{outside of research}
% \begin{itemize}
%   \item working experience as a TA
%   \item president of Student survey $\rightarrow$ leadership award
%   \item co-working capability
% \end{itemize}

% \section{Why Princeton?}
% \begin{itemize}
%   \item Future dream
%   \item what wanna do
%   \item reson why I chose this school
% \end{itemize}

\end{document}
