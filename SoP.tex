\documentclass[a4]{article}
\usepackage[letterpaper,margin=1in]{geometry}
\title{Statement of Purpose: Ver3}
\author{Hiroki Yoshida}
\date{}

\begin{document}
\maketitle

My research interest is strongly focused on the theoretical study of physics, especially condensed matter physics. Through research on nonequilibrium statistical mechanics, I realized that the theoretical study very much fits me. Meanwhile, I became more interested in topics which is more relevant to real-world physics. For that point, the field of topological phases of matter was very attractive. Prof. Bernevig in Princeton is one of the most outstanding researchers in the world and hence, learning at Princeton is the best choice for me.\par

\section{learnings beyond standard carriculum}
I enrolled in the university by the award for the presentation of my experimental research results on photocatalytic activity of TiO$_2$ which I had been doing throughout high school. At that time, my interest was mainly phenomenon itself and had no preference in theory. After taking several classes, it gradually shifted to the theoretical studies in physics. Since our university puts limitations on the number of credits that can be taken per year, I could not take advanced classes officially. However, my enthusiasm could not be stoped and I personally asked teachers for permission to attend courses beyond the curriculum. As a result, I had finished learning most of the physics courses untill the middle of the third year.\par

% especially this block has long long sentences.
The most striking event to make me dive into the world of theoretical physics is the research on nonequilibrium statistical mechanics under Specially appointed Prof.Takahashi in Tokyo Tech. Though it is completely irrelevant with the standard physics curriculum, I had time after taking all the courses and was curious about how research in general is conducted, I personally asked him to do research in collaboration. Ever since August 2020 until now, we are studying the dynamical Lee-Yang zeros of the simple two-level system. Lee-Yang zeros are zeros of the partition function of a system with complex parameters. Though it is not present in the real-valued physical parameters, it is very useful to determine phase transitions in the thermodynamic limit and also recently experimentally utilized in a system where physical parameters can effectively be considered to be complex. These known results are for static systems without time dependence. Therefore, we are currently tackling the hetherto unknown periodically driven cases. In time-dependent cases, the situation changes drastically and determination of the zeros itself becomes very difficult. I mainly do numerical calculations by Python and c++ and also formulated some general formulas for current distribution functions using dynamical Lee-Yang zeros. This research is still in progress and the results are not yet published, but I could learn a lot of hardships in theoretical studies. Based on this 15-month research experience, I realized how much  perseverance I have toward research in physics and also how I am in love with theoretical physics. Combining it with the original interests on phenomena, I decided to specialize in condensed matter theory.\par

\section{current research}
I am investigating the topological phases of matter under the supervision of Prof.Murakami in Tokyo Tech now. My original theme was theoretical understanding of the photovoltaic effect in a van der Waals interface, which was experimentally observed recently. This effect is caused by the breaking of symmetry by stacking two sheets of materials with different rotational symmetries. This "symmetry breaking by symmetries" itself is not only an interesting idea from the theoretical point of view but also very important for future applications because of the variety of symmetries we can think of.  As like Moir\'{e} crystals, lack of periodicity makes understanding this phenomenon difficult. I am trying to construct a simple model which minimally captures the essence. Until graduation, I want to further understand the mechanism and hopefully publish the results. \par

Thanks to my enthusiasm to gain as much research experience as possible, I recently began another research too in parallel with this theme. It is about the newly manufactured hexagonal boron nitride sheet. By the Brillouin zone folding from that of the Graphene, the Dirac cones get gapped and localized states appear in finite systems. I constructed a tight binding hamiltonian of this lattice and confirmed the existence of the edge modes by myself so far. This work is also still in progress but through this research, I acquired the ability to construct a tight-binding model for the desired system in both finite and infinite systems and analyze it analytically and numerically. This would be very helpful in research in graduate school and beyond.\par
% \begin{itemize}
%   \item topological band theory
%   \item constructing tight binding hamiltonian
%   \item optical response
%   \item van der waals heterostructure
%   \item Why this field of study
% \end{itemize}


\section{future plan}
After getting a Ph.D. degree, I want to get an academic position at a university. As a preparation for that, I actively participated in some activities apart from physics researches. One of them is the university official survey team called "student survey". I was the president of that team for more than 2 years and learned how to manage people in a team. For my hard work and prominent leadership, I received a "student leadership award 2021". This leadership and collaborative ability will be my strength in the research field too.\par

Princeton is my top destination for graduate school research. The main reason why I chose Princeton is that Princeton physics is highly concentrated on educating independent researchers. Of course, research activities under the legendary teachers in my field such as Prof.Bernevig or Prof.Haldane will be an extraordinary opportunity and I will definitely make the most of it. However, to be an independent researcher who can establish new fields of study , being satisfied with making use of given chances is not enough. Proactive research collaboration with young researchers in a good environment is essential. For this in my mind, Princeton is the best platform to do graduate school research.


% \begin{itemize}
%   \item study in advance
%   \item research on dynamical lee-yang zeros
%   \item enthusiasm to theoretical research
% \end{itemize}

% \section{outside of research}
% \begin{itemize}
%   \item working experience as a TA
%   \item president of Student survey $\rightarrow$ leadership award
%   \item co-working capability
% \end{itemize}

% \section{Why Princeton?}
% \begin{itemize}
%   \item Future dream
%   \item what wanna do
%   \item reson why I chose this school
% \end{itemize}

\end{document}
